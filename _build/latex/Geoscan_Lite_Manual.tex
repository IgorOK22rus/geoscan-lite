%% Generated by Sphinx.
\def\sphinxdocclass{report}
\documentclass[a4paper,10pt,russian,openany]{sphinxmanual}
\ifdefined\pdfpxdimen
   \let\sphinxpxdimen\pdfpxdimen\else\newdimen\sphinxpxdimen
\fi \sphinxpxdimen=.75bp\relax

\PassOptionsToPackage{warn}{textcomp}
\usepackage[utf8]{inputenc}
\ifdefined\DeclareUnicodeCharacter
% support both utf8 and utf8x syntaxes
  \ifdefined\DeclareUnicodeCharacterAsOptional
    \def\sphinxDUC#1{\DeclareUnicodeCharacter{"#1}}
  \else
    \let\sphinxDUC\DeclareUnicodeCharacter
  \fi
  \sphinxDUC{00A0}{\nobreakspace}
  \sphinxDUC{2500}{\sphinxunichar{2500}}
  \sphinxDUC{2502}{\sphinxunichar{2502}}
  \sphinxDUC{2514}{\sphinxunichar{2514}}
  \sphinxDUC{251C}{\sphinxunichar{251C}}
  \sphinxDUC{2572}{\textbackslash}
\fi
\usepackage{cmap}
\usepackage[T1]{fontenc}
\usepackage{amsmath,amssymb,amstext}
\usepackage{babel}






\usepackage[Sonny]{fncychap}
\ChNameVar{\Large\normalfont\sffamily}
\ChTitleVar{\Large\normalfont\sffamily}
\usepackage{sphinx}

\fvset{fontsize=\small}
\usepackage{geometry}

% Include hyperref last.
\usepackage{hyperref}
% Fix anchor placement for figures with captions.
\usepackage{hypcap}% it must be loaded after hyperref.
% Set up styles of URL: it should be placed after hyperref.
\urlstyle{same}
\addto\captionsrussian{\renewcommand{\contentsname}{Содержание:}}

\usepackage{sphinxmessages}
\setcounter{tocdepth}{1}


    \usepackage{setspace}
    \usepackage{fontspec}
    \setmainfont[Ligatures=TeX]{Georgia}
    \setsansfont[Ligatures=TeX]{Arial}
    

\title{}
\date{}
\release{}
\author{}
\newcommand{\sphinxlogo}{\sphinxincludegraphics{logo_latex.png}\par}
\renewcommand{\releasename}{}
\makeindex
\begin{document}

\ifdefined\shorthandoff
  \ifnum\catcode`\=\string=\active\shorthandoff{=}\fi
  \ifnum\catcode`\"=\active\shorthandoff{"}\fi
\fi

\pagestyle{empty}
\sphinxmaketitle
\pagestyle{plain}
\sphinxtableofcontents
\pagestyle{normal}
\phantomsection\label{\detokenize{index::doc}}



\chapter{Общие сведения}
\label{\detokenize{reference:id1}}\label{\detokenize{reference::doc}}
\noindent{\hspace*{\fill}\sphinxincludegraphics{{101}.png}\hspace*{\fill}}

Геоскан Lite \textendash{} компактный беспилотный комплекс для аэрофотосъемки.

\sphinxstylestrong{Назначение}

Комплекс Геоскан Lite предназначен для получения геопривязанных фотографий объектов и автоматической площадной аэрофотосъемки.

\sphinxstylestrong{Область применения}

Полученные с использованием комплекса материалы могут использоваться для:
\begin{itemize}
\item {} 
оценки состояния местности;

\item {} 
выполнения аэрофотосъемки местности с фиксацией моментов фотографирования для получения геопривязанных фотографий;

\item {} 
создания ортофотопланов и цифровых моделей местности по материалам аэрофотосъемки;

\item {} 
создания карт высот;

\item {} 
создания 3D-моделей местности;

\item {} 
вычисления объемов пород в карьерах и насыпных объектах.

\end{itemize}


\section{Комплект поставки}
\label{\detokenize{reference:id2}}\begin{itemize}
\item {} 
Беспилотное воздушное судно (БВС) Геоскан Lite

\item {} 
Транспортировочная сумка БВС

\item {} 
Пусковая установка в транспортировочной сумке

\item {} 
Модернизированная фотокамера

\item {} 
Аккумуляторная батарея (АКБ) LiPo 14,8 В

\item {} 
Модем КРЛ с антенной

\item {} 
Зарядное устройство для АКБ

\item {} 
ПО для планирования полетного задания Geoscan Planner

\item {} 
Раскладная подставка для сборки планера

\item {} 
Комплект запасных частей:
\begin{itemize}
\item {} 
киль - 2 шт;

\item {} 
штырь креплекия консолей крыла (460 мм);

\item {} 
комплект резиновых жгутов для катапульты;

\item {} 
лопасти воздушного винта 10x8;

\item {} 
ключ рожковый;

\item {} 
шомпол приемника воздушного давления;

\item {} 
приемник воздушного давления;

\item {} 
силиконовая трубка;

\item {} 
нож универсальный.

\end{itemize}

\end{itemize}


\section{Технические характеристики}
\label{\detokenize{reference:id3}}

\begin{savenotes}\sphinxattablestart
\centering
\begin{tabulary}{\linewidth}[t]{|T|T|}
\hline

Тип БВС
&
летающее крыло
\\
\hline
Скорость полета (воздушная)
&
64 - 130 км/ч
\\
\hline
Максимальная взлетная масса
&
3,1 кг
\\
\hline
Максимальная масса полезной нагрузки
&
0,8 кг
\\
\hline
Площадь фотосъемки за один полет
&
до 9 км\(\sp{\text{2}}\)
\\
\hline
Допустимая скорость ветра
&
до 12 м/с
\\
\hline
Размах крыла
&
1,38 м
\\
\hline
Минимальная безопасная высота полета
&
100 м
\\
\hline
Максимальная высота полета
&
4000 м
\\
\hline
Двигатель
&
электрический, бесколлекторный
\\
\hline
Аккумуляторная батарея
&
LiPo 14,8 В
\\
\hline
Время подготовки к взлету
&
10 мин
\\
\hline
Продолжительность полета
&
до 60 мин
\\
\hline
Максимальная протяженность маршрута
&
70 км
\\
\hline
Посадка
&
на парашюте, в автоматическом режиме
\\
\hline
Рабочий диапазон температур:
&
от -20 до +40 °С
\\
\hline
\end{tabulary}
\par
\sphinxattableend\end{savenotes}


\section{Обслуживание}
\label{\detokenize{reference:id4}}
После каждого полета осматривайте БВС на предмет повреждений.

В случае повреждения лопастей винта или килей вы можете заменить их самостоятельно, используя запасные части и инструменты из комплекта.

При выявлении конструкционных повреждений планера или систем необходимо обратиться в \sphinxhref{https://www.geoscan.aero/ru/support}{службу технической поддержки}.

После выполнения 80 полетов рекомендуется отправить БВС на завод-изготовитель для проверки и технического обслуживания.


\section{Хранение}
\label{\detokenize{reference:id6}}
Комплекс Геоскан Lite (без аккумуляторных батарей) и пусковую установку рекомендуется хранить в транспортировочных сумках в сухих помещениях при температуре от 5 до 25 °С и относительной влажности не более 80\%, без конденсации. Срок хранения - 2 года.

Аккумуляторные батареи хранить в сухом прохладном месте, исключающем воздействие прямых солнечных лучей, при температуре от 5 до 25 °С и относительной влажности не более 80\%, без конденсации. Оптимальная температура - от 5 до 10 °С. Оптимальный уровень напряжения АКБ при хранении: 15,4 В (подробнее см. {\hyperref[\detokenize{charger::doc}]{\sphinxcrossref{\DUrole{doc}{Зарядое устройство и АКБ}}}}). Срок хранения - 1 год.


\chapter{Правила безопасности}
\label{\detokenize{precautions:id1}}\label{\detokenize{precautions::doc}}
\sphinxstylestrong{БВС Геоскан Lite является источником повышенной опасности. При проведении полетов необходимо соблюдать следующие правила:}
\begin{itemize}
\item {} 
к запуску и техническому обслуживанию беспилотного воздушного судна допускаются лица, прошедшие обучение согласно «Плану теоретической и практической подготовки оператора по управлению беспилотным комплексом Геоскан Lite»;

\item {} 
исполняйте письменные рекомендации и указания поставщика и (или) производителя по использованию оборудования, отраженные в настоящем руководстве и получаемые непосредственно в период эксплуатации;

\item {} 
при планировании маршрута необходимо изучить район полета и убедиться, что планируемая траектория полета проходит не менее чем на 100 м выше элементов рельефа и высотных объектов (вышек, труб, опор ЛЭП и т.п.);

\item {} 
при выборе направления запуска необходимо убедиться, что в секторе \(\pm\)30° относительно направления взлета на расстоянии L от точки старта отсутствуют объекты высотой 0,2L;

\item {} 
не производить запуск БВС при обнаружении какой-либо неисправности комплекса;

\item {} 
до натяжения резиновых жгутов катапульты убедитесь, что установлен предохранительный штифт катапульты. Не вынимайте штифт до момента запуска;

\item {} 
после подключения батареи питания БВС запрещается находиться в плоскости вращения воздушного винта планера;

\item {} 
не превышайте эксплуатационные ограничения массы БВС, высоты и длительности полета;

\item {} 
не осуществляйте запуск и полет БВС вблизи радиопередающих устройств высокой мощности;

\item {} 
не допускайте посторонних лиц в зону запуска БВС, особенно в направлении взлета;

\item {} 
при планировании точки посадки БВС убедитесь, что БВС приземлится вне автомобильных дорог, линий электропередач, водоёмов, мест скопления людей. Примите во внимание возможный снос при спуске на парашюте;

\item {} 
избегайте полетов над густонаселенными районами.

\end{itemize}

\sphinxstylestrong{При работе с БВС необходимо также соблюдать следующие меры предосторожности:}
\begin{itemize}
\item {} 
не допускаются сборка, разборка БВС с включенным питанием;

\item {} 
запрещается находиться в непосредственной близости от воздушного винта при включенном питании БВС;

\item {} 
установка и снятие воздушного винта допускаются только при выключенном питании БВС;

\item {} 
не допускайте короткого замыкания контактов аккумуляторной батареи.

\end{itemize}

\sphinxstylestrong{Во избежание повреждения деталей комплекса:}
\begin{itemize}
\item {} 
транспортируйте БВС только в заводской сумке;

\item {} 
не отклоняйте элевоны БВС вручную;

\item {} 
при переноске БВС в районе старта/посадки следует держать его за фюзеляж;

\item {} 
снимайте крышку объектива фотоаппарата только на время проведения аэрофотосъемочных работ;

\item {} 
при попадании БВС в воду необходимо немедленно отключить АКБ и просушить все элементы конструкции, провода, разъемы, электронное оборудование;

\item {} 
запрещается вносить изменения в конструкцию БВС и пусковой установки.

\end{itemize}


\section{Эксплуатационные ограничения}
\label{\detokenize{precautions:id2}}\begin{itemize}
\item {} 
Рабочий диапазон температур: от -20 до 40 °С

\item {} 
Максимальная сила ветра: 12 м/с

\end{itemize}

Аэрофотосъемочный комплекс не предназначен для полетов во время дождя, снега и прочих атмосферных осадков.

Комплекс не способен выполнять полеты ниже высоты точки старта.

В горной местности старт необходимо осуществлять в низшей точке, чтобы весь маршрут лежал выше точки старта.


\chapter{БВС}
\label{\detokenize{uav:id1}}\label{\detokenize{uav::doc}}

\section{Узлы и детали}
\label{\detokenize{uav:id2}}
\noindent{\hspace*{\fill}\sphinxincludegraphics[width=600\sphinxpxdimen]{{Uav}.png}\hspace*{\fill}}


\section{Сборка}
\label{\detokenize{uav:id3}}\begin{enumerate}
\def\theenumi{\arabic{enumi}}
\def\labelenumi{\theenumi )}
\makeatletter\def\p@enumii{\p@enumi \theenumi )}\makeatother
\item {} 
Извлеките консоли крыла и фюзеляж из транспортировочной сумки.

\item {} 
Уложите парашют в соответствующий отсек в фюзеляже (см. \sphinxstylestrong{Парашютная система}).

\item {} 
Снимите верхнюю крышку фюзеляжа. Для этого отстегните резиновые фиксаторы на носовой части фюзеляжа, затем извлеките заднюю часть крышки из пазов.

\end{enumerate}

\begin{figure}[H]
\centering
\capstart

\noindent\sphinxincludegraphics[width=400\sphinxpxdimen]{{asmbl1}.png}
\caption{Снятие крышки фюзеляжа}\label{\detokenize{uav:id6}}\end{figure}
\begin{enumerate}
\def\theenumi{\arabic{enumi}}
\def\labelenumi{\theenumi )}
\makeatletter\def\p@enumii{\p@enumi \theenumi )}\makeatother
\setcounter{enumi}{3}
\item {} 
Вставьте длинный соединительный штырь в трубку фюзеляжа.

\end{enumerate}

\begin{figure}[H]
\centering
\capstart

\noindent\sphinxincludegraphics[width=300\sphinxpxdimen]{{asmbl2}.png}
\caption{Вставка соединительного штыря}\label{\detokenize{uav:id7}}\end{figure}
\begin{enumerate}
\def\theenumi{\arabic{enumi}}
\def\labelenumi{\theenumi )}
\makeatletter\def\p@enumii{\p@enumi \theenumi )}\makeatother
\setcounter{enumi}{4}
\item {} 
Наденьте консоль крыла на соединительные штыри и продвиньте к фюзеляжу так, чтобы она зашла в паз на фюзеляже и уперлась в ограничители.
Аналогично установите другую консоль.

\end{enumerate}

\begin{figure}[H]
\centering
\capstart

\noindent\sphinxincludegraphics[width=325\sphinxpxdimen]{{asmbl3}.png}
\caption{Установка консоли крыла}\label{\detokenize{uav:id8}}\end{figure}
\begin{enumerate}
\def\theenumi{\arabic{enumi}}
\def\labelenumi{\theenumi )}
\makeatletter\def\p@enumii{\p@enumi \theenumi )}\makeatother
\setcounter{enumi}{5}
\item {} 
Установите кили в консоли. Убедитесь, что кили зафиксированы магнитами.

\end{enumerate}

\begin{figure}[H]
\centering
\capstart

\noindent\sphinxincludegraphics[width=400\sphinxpxdimen]{{asmbl4}.png}
\caption{Установка килей}\label{\detokenize{uav:id9}}\end{figure}
\begin{enumerate}
\def\theenumi{\arabic{enumi}}
\def\labelenumi{\theenumi )}
\makeatletter\def\p@enumii{\p@enumi \theenumi )}\makeatother
\setcounter{enumi}{6}
\item {} 
Подключите разъемы кабельных сборок консолей в соответствующие гнезда автопилота.

\end{enumerate}

\begin{figure}[H]
\centering
\capstart

\noindent\sphinxincludegraphics[width=400\sphinxpxdimen]{{asmbl5}.png}
\caption{Подключение кабелей консолей}\label{\detokenize{uav:id10}}\end{figure}
\begin{enumerate}
\def\theenumi{\arabic{enumi}}
\def\labelenumi{\theenumi )}
\makeatletter\def\p@enumii{\p@enumi \theenumi )}\makeatother
\setcounter{enumi}{7}
\item {} 
Извлеките карты памяти из автопилота и фотоаппарата, отформатируйте их и установите на место.

\item {} 
Установите АКБ и закрепите с помощью текстильной застежки.

\item {} 
Подключите разъем питания.

\end{enumerate}

\begin{figure}[H]
\centering
\capstart

\noindent\sphinxincludegraphics[width=500\sphinxpxdimen]{{asmbl6}.png}
\caption{Установка карты microSD. Установка АКБ. Подключение питания.}\label{\detokenize{uav:id11}}\end{figure}
\begin{enumerate}
\def\theenumi{\arabic{enumi}}
\def\labelenumi{\theenumi )}
\makeatletter\def\p@enumii{\p@enumi \theenumi )}\makeatother
\setcounter{enumi}{10}
\item {} 
Настройте фотоаппарат (см. {\hyperref[\detokenize{camera::doc}]{\sphinxcrossref{\DUrole{doc}{Настройка фотокамеры}}}}). Установите фотоаппарат в ложемент.

\item {} 
Закройте верхнюю крышку фюзеляжа. Для этого сначала зафиксируйте заднюю часть в пазах, затем закрепите крышку с помощью резиновых фиксаторов. Следите за тем, чтобы фиксирующие штыри на верхней крышке фюзеляжа углубились в соответствующие выемки в консолях крыла.

\end{enumerate}

\begin{figure}[H]
\centering
\capstart

\noindent\sphinxincludegraphics[width=400\sphinxpxdimen]{{asmbl7}.png}
\caption{Закрытие крышки фюзеляжа}\label{\detokenize{uav:id12}}\end{figure}

БВС готово к прохождению предстартовой подготовки.


\section{Парашютная система}
\label{\detokenize{uav:id4}}
Составные части парашютной системы:

\begin{figure}[H]
\centering
\capstart

\noindent\sphinxincludegraphics[width=350\sphinxpxdimen]{{para1}.png}
\caption{1 \textendash{} крышка парашютного отсека, выполняющая функцию вытяжного парашюта;
2 \textendash{} купол парашюта;
3 \textendash{} карманы для укладки строп;
4 \textendash{} стропы;
5 \textendash{} стопорное кольцо, предназначенное для замедления раскрытия купола;
6 \textendash{} длинный фал;
7 \textendash{} короткий фал с кольцом системы отцепа.}\label{\detokenize{uav:id13}}\end{figure}

\begin{sphinxadmonition}{attention}{Внимание:}
Перед укладкой парашюта убедитесь, что купол парашюта, стропы и крепления строп к куполу не повреждены, не имеют надрывов и разрезов. Купол и стропы должны быть сухими, на них не должно быть бензиновых или масляных пятен. В случае, если укладка парашюта производилась более чем за 10 суток до вылета, или комплекс перевозился авиационным транспортом, разверните парашют и уложите его заново.
\end{sphinxadmonition}

\sphinxstylestrong{Порядок укладки парашюта}
\begin{enumerate}
\def\theenumi{\arabic{enumi}}
\def\labelenumi{\theenumi )}
\makeatletter\def\p@enumii{\p@enumi \theenumi )}\makeatother
\item {} 
Проверьте состояние парашюта.

\item {} 
Убедитесь, что стропы не запутаны, а крышка парашютного отсека находится снаружи купола.

\item {} 
Следите, чтобы в процессе укладки парашюта стропы не путались.

\item {} 
Контролируйте положение карманов для строп, они должны оставаться на наружной стороне сложенного купола.

\item {} 
Расправьте купол и, совмещая вырезы друг с другом, сложите его пополам.

\end{enumerate}

\begin{figure}[H]
\centering
\capstart

\noindent\sphinxincludegraphics[width=500\sphinxpxdimen]{{para2}.png}
\caption{Складывание купола пополам}\label{\detokenize{uav:id14}}\end{figure}
\begin{enumerate}
\def\theenumi{\arabic{enumi}}
\def\labelenumi{\theenumi )}
\makeatletter\def\p@enumii{\p@enumi \theenumi )}\makeatother
\setcounter{enumi}{5}
\item {} 
Сложите купол пополам второй раз и выровняйте края.

\end{enumerate}

\begin{figure}[H]
\centering
\capstart

\noindent\sphinxincludegraphics[width=475\sphinxpxdimen]{{para3}.png}
\caption{Складывание купола пополам второй раз}\label{\detokenize{uav:id15}}\end{figure}
\begin{enumerate}
\def\theenumi{\arabic{enumi}}
\def\labelenumi{\theenumi )}
\makeatletter\def\p@enumii{\p@enumi \theenumi )}\makeatother
\setcounter{enumi}{6}
\item {} 
В результате стропы должны собраться в 4 пучка по 4 стропы в каждом.

\end{enumerate}

\begin{figure}[H]
\centering
\capstart

\noindent\sphinxincludegraphics[height=350\sphinxpxdimen]{{para4}.png}
\caption{Результат}\label{\detokenize{uav:id16}}\end{figure}
\begin{enumerate}
\def\theenumi{\arabic{enumi}}
\def\labelenumi{\theenumi )}
\makeatletter\def\p@enumii{\p@enumi \theenumi )}\makeatother
\setcounter{enumi}{7}
\item {} 
Еще раз сложите купол пополам так, чтобы карманы для укладки строп оказались снаружи.

\end{enumerate}

\begin{figure}[H]
\centering
\capstart

\noindent\sphinxincludegraphics[width=150\sphinxpxdimen]{{para5}.png}
\caption{Складывание купола карманами наружу}\label{\detokenize{uav:id17}}\end{figure}
\begin{enumerate}
\def\theenumi{\arabic{enumi}}
\def\labelenumi{\theenumi )}
\makeatletter\def\p@enumii{\p@enumi \theenumi )}\makeatother
\setcounter{enumi}{8}
\item {} 
Аккуратно сложите купол «гармошкой», как показано на рисунке.

\end{enumerate}

\begin{figure}[H]
\centering
\capstart

\noindent\sphinxincludegraphics[width=400\sphinxpxdimen]{{para6}.png}
\caption{Складывание купола «гармошкой»}\label{\detokenize{uav:id18}}\end{figure}

Убедитесь, что стропы не перекручены и не перехлестнуты. При необходимости расправьте стропы.
\begin{enumerate}
\def\theenumi{\arabic{enumi}}
\def\labelenumi{\theenumi )}
\makeatletter\def\p@enumii{\p@enumi \theenumi )}\makeatother
\setcounter{enumi}{9}
\item {} 
Уложите стропы в карман. Для этого отмерьте длину пучка строп, превышающую глубину кармана. Сложите пучок пополам и протяните в карман так, чтобы перегиб пучка на несколько сантиметров выступал с противоположной стороны кармана. При необходимости устраните слабину строп у края купола, подтянув их за перегиб с противоположной стороны пучка.

\end{enumerate}

\begin{figure}[H]
\centering
\capstart

\noindent\sphinxincludegraphics[width=275\sphinxpxdimen]{{para8}.png}
\caption{Укладка строп в карман}\label{\detokenize{uav:id19}}\end{figure}
\begin{enumerate}
\def\theenumi{\arabic{enumi}}
\def\labelenumi{\theenumi )}
\makeatletter\def\p@enumii{\p@enumi \theenumi )}\makeatother
\setcounter{enumi}{10}
\item {} 
Передвиньте стопорное кольцо к куполу. Заправьте оставшуюся часть строп в свободный карман купола. Для этого сложите оставшуюся часть пополам и проденьте в карман так, чтобы оплетка на узле соединения строп с фалом касалась кармана.

\end{enumerate}

\begin{figure}[H]
\centering
\capstart

\noindent\sphinxincludegraphics[width=200\sphinxpxdimen]{{para7}.png}
\caption{Протягивание строп через карман}\label{\detokenize{uav:id20}}\end{figure}

\begin{figure}[H]
\centering
\capstart

\noindent\sphinxincludegraphics[width=275\sphinxpxdimen]{{para9}.png}
\caption{Результат}\label{\detokenize{uav:id21}}\end{figure}

После укладки оставшейся части строп в карман купола кольцо должно находиться между витками строп.
\begin{enumerate}
\def\theenumi{\arabic{enumi}}
\def\labelenumi{\theenumi )}
\makeatletter\def\p@enumii{\p@enumi \theenumi )}\makeatother
\setcounter{enumi}{11}
\item {} 
Сложите купол «гармошкой», как показано на рисунке.

\end{enumerate}

\begin{figure}[H]
\centering
\capstart

\noindent\sphinxincludegraphics[width=400\sphinxpxdimen]{{para10}.png}
\caption{Складывание купола «гармошкой»}\label{\detokenize{uav:id22}}\end{figure}


\section{Установка парашюта в БВС}
\label{\detokenize{uav:id5}}\begin{enumerate}
\def\theenumi{\arabic{enumi}}
\def\labelenumi{\theenumi )}
\makeatletter\def\p@enumii{\p@enumi \theenumi )}\makeatother
\item {} 
Переверните БВС так, чтобы парашютный отсек оказался сверху.

\item {} 
Возьмите в руки карабин подвеса и расправьте тросики.

\item {} 
Проденьте конец короткого фала парашюта в карабин подвеса со стороны передней части БВС.

\end{enumerate}

\begin{figure}[H]
\centering
\capstart

\noindent\sphinxincludegraphics[width=325\sphinxpxdimen]{{param1}.png}
\caption{Продевание конца короткого фала в карабин}\label{\detokenize{uav:id23}}\end{figure}
\begin{enumerate}
\def\theenumi{\arabic{enumi}}
\def\labelenumi{\theenumi )}
\makeatletter\def\p@enumii{\p@enumi \theenumi )}\makeatother
\setcounter{enumi}{3}
\item {} 
Проденьте конец короткого фала через кольцо системы отцепа со стороны двигателя.

\end{enumerate}

\begin{figure}[H]
\centering
\capstart

\noindent\sphinxincludegraphics[width=300\sphinxpxdimen]{{param2}.png}
\caption{Продевание конца короткого фала через кольцо системы отцепа}\label{\detokenize{uav:id24}}\end{figure}
\begin{enumerate}
\def\theenumi{\arabic{enumi}}
\def\labelenumi{\theenumi )}
\makeatletter\def\p@enumii{\p@enumi \theenumi )}\makeatother
\setcounter{enumi}{4}
\item {} 
Пропустите конец короткого фала через отверстие системы отцепа и защелкните петлю на его конце между половинками замка.

\end{enumerate}

\begin{figure}[H]
\centering
\capstart

\noindent\sphinxincludegraphics[width=400\sphinxpxdimen]{{param3}.png}
\caption{Защелкивание петли в замке}\label{\detokenize{uav:id25}}\end{figure}

\begin{sphinxadmonition}{attention}{Внимание:}
Будьте внимательны при переноске и установке БВС на пусковую установку. Случайное нажатие на язычок системы отцепа может привести к открытию замка системы отцепа и, как следствие, к преждевременному отделению парашюта при посадке БВС.
\end{sphinxadmonition}

\begin{sphinxadmonition}{note}{Примечание:}
Убедитесь, что замок системы отцепа надежно защелкнут. Для этого поднимите БВС за фал парашюта и сделайте несколько коротких резких рывков вверх.
\end{sphinxadmonition}
\begin{enumerate}
\def\theenumi{\arabic{enumi}}
\def\labelenumi{\theenumi )}
\makeatletter\def\p@enumii{\p@enumi \theenumi )}\makeatother
\setcounter{enumi}{5}
\item {} 
Аккуратно уложите тросики подвесной системы и фал на дно парашютного отсека.

\end{enumerate}

\begin{figure}[H]
\centering
\capstart

\noindent\sphinxincludegraphics[width=275\sphinxpxdimen]{{param4}.png}
\caption{Укладка тросиков подвесной системы}\label{\detokenize{uav:id26}}\end{figure}
\begin{enumerate}
\def\theenumi{\arabic{enumi}}
\def\labelenumi{\theenumi )}
\makeatletter\def\p@enumii{\p@enumi \theenumi )}\makeatother
\setcounter{enumi}{6}
\item {} 
Уложите сверху сложенный парашют так, чтобы его основание с кольцом оказалось на дне парашютного отсека.

\end{enumerate}

\begin{figure}[H]
\centering
\capstart

\noindent\sphinxincludegraphics[width=325\sphinxpxdimen]{{param5}.png}
\caption{Укладка парашюта}\label{\detokenize{uav:id27}}\end{figure}
\begin{enumerate}
\def\theenumi{\arabic{enumi}}
\def\labelenumi{\theenumi )}
\makeatletter\def\p@enumii{\p@enumi \theenumi )}\makeatother
\setcounter{enumi}{7}
\item {} 
Вставьте выступ на задней части крышки парашютного отсека в паз на корпусе БВС и опустите крышку. При необходимости заправьте края купола или фал под крышку.

\end{enumerate}

\begin{sphinxadmonition}{attention}{Внимание:}
Проверьте, что крышка парашютного отсека свободно открывается и закрывается. Для этого поверните поводок машинки отцепа в сторону и приподнимите крышку вверх. Стропа крепления крышки к парашюту не должна западать или цепляться за подкрепляющий штырь крышки. Убедитесь, что крышка свободно открывается и закрывается, а купол сложенного парашюта не попадает в места прилегания крышки к фюзеляжу.
\end{sphinxadmonition}
\begin{enumerate}
\def\theenumi{\arabic{enumi}}
\def\labelenumi{\theenumi )}
\makeatletter\def\p@enumii{\p@enumi \theenumi )}\makeatother
\setcounter{enumi}{8}
\item {} 
Закройте крышку, прижав ее и сдвинув поводок машинки отцепа.

\end{enumerate}

\begin{figure}[H]
\centering
\capstart

\noindent\sphinxincludegraphics[width=400\sphinxpxdimen]{{param6}.png}
\caption{Фиксация крышки парашютного отсека}\label{\detokenize{uav:id28}}\end{figure}

\begin{sphinxadmonition}{attention}{Внимание:}
Категорически запрещается поворачивать поводок машинки парашютного отсека рукой при включенном электропитании БВС.
\end{sphinxadmonition}


\chapter{Настройка НСУ}
\label{\detokenize{nsu:id1}}\label{\detokenize{nsu::doc}}
Данное руководство предполагает развертывание наземной станции управления на базе ноутбука (не входит в базовый комплект поставки).

\sphinxstylestrong{Минимальные системные требования}


\begin{savenotes}\sphinxattablestart
\centering
\begin{tabulary}{\linewidth}[t]{|T|T|}
\hline

Операционная система
&
MS Windows 7,8,10
\\
\hline
Процессор
&
Intel Core i3
\\
\hline
Оперативная память
&
4 Гб
\\
\hline
Тип видеокарты
&
Дискретная
\\
\hline
Чипсет видеокарты
&
Nvidia GeForce GT620M, GT630M, 710M, GT720M; AMD Radeon HD 7670M
\\
\hline
\end{tabulary}
\par
\sphinxattableend\end{savenotes}

\sphinxstylestrong{Рекомендуемые системные требования}


\begin{savenotes}\sphinxattablestart
\centering
\begin{tabulary}{\linewidth}[t]{|T|T|}
\hline

Операционная система
&
MS Windows 7,8,10
\\
\hline
Процессор
&
Intel Core i5, i7
\\
\hline
Оперативная память
&
8 Гб
\\
\hline
Тип видеокарты
&
Дискретная
\\
\hline
Чипсет видеокарты
&
Nvidia GeForce GT645M, GT745M, 845M, GT720M, 940M и выше
\\
\hline
\end{tabulary}
\par
\sphinxattableend\end{savenotes}


\section{Комплект}
\label{\detokenize{nsu:id2}}\begin{itemize}
\item {} 
USB flash-накопитель с ПО

\item {} 
Радиомодем

\item {} 
Антенна радиомодема

\item {} 
Стойка радиомодема

\end{itemize}

\newpage

\section{Развертывание}
\label{\detokenize{nsu:id3}}\begin{enumerate}
\def\theenumi{\arabic{enumi}}
\def\labelenumi{\theenumi )}
\makeatletter\def\p@enumii{\p@enumi \theenumi )}\makeatother
\item {} 
Извлеките радиомодем и антенну из транспортировочной сумки БВС.

\item {} 
Закрутите антенну в разьем антенны.

\item {} 
Установите стойку для радиомодема и закрепите на ней модем, вставив крепежные винты в прорезь в верхней части стойки.

\end{enumerate}

Стойку можно установить, заглубив наконечник в грунт.
\begin{enumerate}
\def\theenumi{\arabic{enumi}}
\def\labelenumi{\theenumi )}
\makeatletter\def\p@enumii{\p@enumi \theenumi )}\makeatother
\setcounter{enumi}{3}
\item {} 
Запустите ноутбук.

\item {} 
Установите ПО MdmDisp и Geoscan Planner с flash-накопителя.

\item {} 
Подключите модем к ноутбуку.

\end{enumerate}

\sphinxstylestrong{Радиомодем должен быть установлен на максимально возможной высоте. Антенна должна быть расположена вертикально.}
\sphinxstylestrong{Не устанавливайте модем внутри автомобиля или помещения.}

\begin{sphinxadmonition}{warning}{Предупреждение:}
Запрещается подключать радиомодем к ноутбуку без присоединенной антенны. Невыполнение данного требования может привести к выходу модема из строя.
\end{sphinxadmonition}


\chapter{Пусковая установка}
\label{\detokenize{catapult:id1}}\label{\detokenize{catapult::doc}}
\noindent{\hspace*{\fill}\sphinxincludegraphics[width=450\sphinxpxdimen]{{catapult_all}.png}\hspace*{\fill}}


\section{Сборка}
\label{\detokenize{catapult:id2}}\begin{enumerate}
\def\theenumi{\arabic{enumi}}
\def\labelenumi{\theenumi )}
\makeatletter\def\p@enumii{\p@enumi \theenumi )}\makeatother
\item {} 
Извлеките детали пусковой установки из транспортировочной сумки.

\item {} 
Разложите опоры передней части пусковой установки. Убедитесь, что они надежно за креплены подпружиненными защелками. Чтобы сложить стойки, оттяните защелки вниз.

\end{enumerate}

\begin{figure}[H]
\centering
\capstart

\noindent\sphinxincludegraphics[width=175\sphinxpxdimen]{{catapult1}.png}
\caption{Установка опор}\label{\detokenize{catapult:id5}}\end{figure}
\begin{enumerate}
\def\theenumi{\arabic{enumi}}
\def\labelenumi{\theenumi )}
\makeatletter\def\p@enumii{\p@enumi \theenumi )}\makeatother
\setcounter{enumi}{2}
\item {} 
Присоедините среднюю часть пусковой установки.

\end{enumerate}

\begin{figure}[H]
\centering
\capstart

\noindent\sphinxincludegraphics[width=320\sphinxpxdimen]{{catapult2}.png}
\caption{Присоединение средней части}\label{\detokenize{catapult:id6}}\end{figure}
\begin{enumerate}
\def\theenumi{\arabic{enumi}}
\def\labelenumi{\theenumi )}
\makeatletter\def\p@enumii{\p@enumi \theenumi )}\makeatother
\setcounter{enumi}{3}
\item {} 
Присоедините заднюю часть.

\end{enumerate}

\begin{figure}[H]
\centering
\capstart

\noindent\sphinxincludegraphics[width=320\sphinxpxdimen]{{catapult3}.png}
\caption{Присоединение задней части}\label{\detokenize{catapult:id7}}\end{figure}
\begin{enumerate}
\def\theenumi{\arabic{enumi}}
\def\labelenumi{\theenumi )}
\makeatletter\def\p@enumii{\p@enumi \theenumi )}\makeatother
\setcounter{enumi}{4}
\item {} 
Вставьте фиксирующий штифт в отверстие в соединении средней и задней частей пусковой установки.

\item {} 
Установите пусковую установку на ровной поверхности так, чтобы запуск БВС происходил против ветра. Убедитесь, что пусковая установка устойчиво стоит на земле, а ее направляющая не имеет крена. При необходимости заглубите одну из опор для выравнивания конструкции.

\end{enumerate}

\begin{sphinxadmonition}{attention}{Внимание:}
Запуск БВС разрешен строго против ветра. Категорически запрещается производить запуск БВС по ветру. Невыполнение данного требования может привезти к падению БВС или к столкновению его с препятствиями, поскольку БВС не сможет набрать высоту.
\end{sphinxadmonition}
\begin{enumerate}
\def\theenumi{\arabic{enumi}}
\def\labelenumi{\theenumi )}
\makeatletter\def\p@enumii{\p@enumi \theenumi )}\makeatother
\setcounter{enumi}{6}
\item {} 
Установите и забейте упорный кол в кронштейн задней части пусковой установки, чтобы предотвратить смещение при запуске БВС.

\end{enumerate}

\begin{sphinxadmonition}{attention}{Внимание:}
При запуске БВС рывок резиновых жгутов приводит к подбрасыванию задней части пусковой установки. Чтобы надежно закрепить пусковую установку, забивайте кол на всю длину, добиваясь полной неподвижности задних опорных стоек пусковой установки. Забивая кол, следите за тем, чтобы не деформировать заднюю часть пусковой установки.
\end{sphinxadmonition}
\begin{enumerate}
\def\theenumi{\arabic{enumi}}
\def\labelenumi{\theenumi )}
\makeatletter\def\p@enumii{\p@enumi \theenumi )}\makeatother
\setcounter{enumi}{7}
\item {} 
Установите каретку на направляющую пусковой установки так, чтобы направляющие каретки скользили по рельсам.

\end{enumerate}

\begin{figure}[H]
\centering
\capstart

\noindent\sphinxincludegraphics[width=400\sphinxpxdimen]{{catapult4}.png}
\caption{Установка каретки}\label{\detokenize{catapult:id8}}\end{figure}

\begin{sphinxadmonition}{attention}{Внимание:}
Убедитесь, что каретка свободно скользит по рельсам, не цепляясь на стыках частей пусковой установки. Осуществляйте проверку свободного движения каретки перед каждым запуском БВС.
\end{sphinxadmonition}
\begin{enumerate}
\def\theenumi{\arabic{enumi}}
\def\labelenumi{\theenumi )}
\makeatletter\def\p@enumii{\p@enumi \theenumi )}\makeatother
\setcounter{enumi}{8}
\item {} 
Отведите каретку вниз до фиксации в замке (должно быть два щелчка).

\item {} 
Вставьте предохранительный штифт в пусковой механизм.

\end{enumerate}

\begin{figure}[H]
\centering
\capstart

\noindent\sphinxincludegraphics[width=300\sphinxpxdimen]{{catapult8}.png}
\caption{Вставка предохранительного штифта}\label{\detokenize{catapult:id9}}\end{figure}

\begin{sphinxadmonition}{attention}{Внимание:}
Неисполнение данного требования может привести к случайному срабатыванию замка.
\end{sphinxadmonition}
\begin{enumerate}
\def\theenumi{\arabic{enumi}}
\def\labelenumi{\theenumi )}
\makeatletter\def\p@enumii{\p@enumi \theenumi )}\makeatother
\setcounter{enumi}{10}
\item {} 
Отпустите стопор лебедки и размотайте натяжной трос.

\item {} 
Возьмите резиновые жгуты, расправьте кольца, убедитесь, что жгуты не перехлестнуты.

\item {} 
Из кольца веревки на конце жгута сделайте петлю и зацепите каретку.

\end{enumerate}

\begin{figure}[H]
\centering
\capstart

\noindent\sphinxincludegraphics[width=250\sphinxpxdimen]{{catapult10}.png}
\caption{Петля на кольце веревки}\label{\detokenize{catapult:id10}}\end{figure}
\begin{enumerate}
\def\theenumi{\arabic{enumi}}
\def\labelenumi{\theenumi )}
\makeatletter\def\p@enumii{\p@enumi \theenumi )}\makeatother
\setcounter{enumi}{13}
\item {} 
Второй конец резинового жгута соедините с концом натяжного троса с помощью карабина. Натяжной трос должен проходить через ролик. Карабин обязательно должен быть замуфтован.

\item {} 
Аналогично присоедините второй резиновый жгут.

\item {} 
Наденьте ручку лебедки на ось и сдвиньте до щелчка подкрепляющего механизма.

\end{enumerate}

\begin{figure}[H]
\centering
\capstart

\noindent\sphinxincludegraphics[width=400\sphinxpxdimen]{{catapult11}.png}
\caption{Установка ручки лебедки}\label{\detokenize{catapult:id11}}\end{figure}


\section{Подготовка пусковой установки к запуску БВС}
\label{\detokenize{catapult:id3}}
\begin{sphinxadmonition}{attention}{Внимание:}
В целях обеспечения безопасности работ на стартовой площадке и увеличения ресурса резиновых жгутов, натягивайте жгуты непосредственно перед стартом после успешного проведения предстартовой подготовки.
\end{sphinxadmonition}
\begin{enumerate}
\def\theenumi{\arabic{enumi}}
\def\labelenumi{\theenumi )}
\makeatletter\def\p@enumii{\p@enumi \theenumi )}\makeatother
\item {} 
Переведите стопор лебедки в положение для натяжения троса.

\item {} 
Вращая ручку лебедки, натяните резиновые жгуты. Особое внимание следует обратить на карабины во время прохождения их через ролики. При попадании жгута между роликом и направляющей пусковой установки или срыве с ролика необходимо прекратить натяжение жгутов. Сорвавшийся жгут нужно уложить на ролик, после чего можно продолжить взводить пусковую установку.

\end{enumerate}

Натяжение необходимо прекратить, когда карабины, за которые зацеплены жгуты, будут напротив маркера «STOP», выгравированного на средней части пусковой установки.

Теперь БВС может быть установлено на пусковую установку.


\section{Правила обращения с резиновыми жгутами}
\label{\detokenize{catapult:id4}}\begin{itemize}
\item {} 
Регулярно проверяйте состояние резиновых жгутов. При обнаружении трещин и потертостей замените поврежденное кольцо на запасное из комплекта ЗИП.

\item {} 
Не держите резиновые жгуты в натянутом состоянии длительное время. Натягивайте жгуты непосредственно перед установкой БВС на пусковую установку.

\item {} 
В теплое время года не допускайте длительного нахождения резиновых жгутов под прямыми солнечными лучами.

\item {} 
В холодное время года не допускайте замерзания резиновых жгутов. Держите их в теплом месте до установки на пусковую установку и натягивайте непосредственно перед запуском. После запуска БВС немедленно снимите жгуты с пусковой установки и уберите в теплое место.

\end{itemize}


\chapter{Настройка фотокамеры}
\label{\detokenize{camera:id1}}\label{\detokenize{camera::doc}}

\section{Настройка фотокамеры Sony A6000}
\label{\detokenize{camera:sony-a6000}}
Перед настройкой ознакомьтесь \sphinxhref{https://www.sony.ru/electronics/support/res/manuals/4532/45320554M.pdf}{инструкцией по эксплуатации фотокамеры} , в которой описано назначение и использование органов управления.

Диск режимов установите в положение \sphinxstylestrong{S} (Приоритет выдержки).
С помощью диска управления установите следующие параметры:


\begin{savenotes}\sphinxattablestart
\centering
\begin{tabulary}{\linewidth}[t]{|T|T|}
\hline

\sphinxstylestrong{выдержка}
&
\sphinxstylestrong{1/1000}
\\
\hline
\sphinxstylestrong{ISO}
&
\sphinxstylestrong{Auto}
\\
\hline
\end{tabulary}
\par
\sphinxattableend\end{savenotes}
\begin{itemize}
\item {} 
В меню фотосъемки \sphinxincludegraphics[width=25\sphinxpxdimen]{{icon_cam}.png} (вкладка 2) установите \sphinxstylestrong{Режим фокусировки} — \sphinxstylestrong{Ручной фокус}.

\end{itemize}

\begin{figure}[H]
\centering
\capstart

\noindent\sphinxincludegraphics[width=300\sphinxpxdimen]{{menu4}.png}
\caption{Установка ручного фокуса для режима фокусировки}\label{\detokenize{camera:id4}}\end{figure}
\begin{itemize}
\item {} 
В меню пользовательских установок \sphinxincludegraphics[width=25\sphinxpxdimen]{{icon_set}.png} (вкладка 1) отключите \sphinxstylestrong{Автоматический просмотр}.

\end{itemize}

\begin{figure}[H]
\centering
\capstart

\noindent\sphinxincludegraphics[width=300\sphinxpxdimen]{{menu5}.png}
\caption{Отключение автоматического просмотра}\label{\detokenize{camera:id5}}\end{figure}
\begin{itemize}
\item {} 
В меню пользовательских установок \sphinxincludegraphics[width=25\sphinxpxdimen]{{icon_set}.png} (вкладка 3) включите \sphinxstylestrong{Спуск без объектива}.

\end{itemize}

\begin{figure}[H]
\centering
\capstart

\noindent\sphinxincludegraphics[width=300\sphinxpxdimen]{{menu6}.png}
\caption{Включение спуска без объектива}\label{\detokenize{camera:id6}}\end{figure}
\begin{itemize}
\item {} 
В меню пользовательских установок (вкладка 6) установите для кнопки \sphinxstylestrong{MOVIE} — \sphinxstylestrong{Только режим видео}.

\end{itemize}

\begin{figure}[H]
\centering
\capstart

\noindent\sphinxincludegraphics[width=300\sphinxpxdimen]{{menu7}.png}
\caption{Установка значения «Только режим видео»}\label{\detokenize{camera:id7}}\end{figure}
\begin{itemize}
\item {} 
В меню настроек \sphinxincludegraphics[width=25\sphinxpxdimen]{{icon_bag}.png} (вкладка 2) установите \sphinxstylestrong{Время начала энергосбережения} - \sphinxstylestrong{30 мин}.

\end{itemize}

\begin{figure}[H]
\centering
\capstart

\noindent\sphinxincludegraphics[width=300\sphinxpxdimen]{{menu8}.png}
\caption{Установка времени начала энергосбережения}\label{\detokenize{camera:id8}}\end{figure}
\begin{itemize}
\item {} 
В меню настроек \sphinxincludegraphics[width=25\sphinxpxdimen]{{icon_bag}.png} (вкладка 5) установите \sphinxstylestrong{Номер файла} - \sphinxstylestrong{Сброс}.

\end{itemize}

\begin{figure}[H]
\centering
\capstart

\noindent\sphinxincludegraphics[width=300\sphinxpxdimen]{{menu9}.png}
\caption{Сброс номера файла}\label{\detokenize{camera:id9}}\end{figure}

\sphinxstylestrong{Форматиравание карты памяти}
\begin{enumerate}
\def\theenumi{\arabic{enumi}}
\def\labelenumi{\theenumi )}
\makeatletter\def\p@enumii{\p@enumi \theenumi )}\makeatother
\item {} 
Последовательно выберите \sphinxstylestrong{MENU \(\rightarrow\)} \sphinxincludegraphics[width=25\sphinxpxdimen]{{icon_bag}.png} \sphinxstylestrong{{[}Настройка{]}} \sphinxstylestrong{\(\rightarrow\) Форматировать}

\end{enumerate}

\begin{sphinxadmonition}{attention}{Внимание:}
Все данные на карте памяти будут удалены!
\end{sphinxadmonition}


\section{Настройка фотокамеры Sony A6000 NIR}
\label{\detokenize{camera:sony-a6000-nir}}
В меню фотосъемки установите следующие параметры:
\begin{itemize}
\item {} 
\sphinxstylestrong{Качество} - \sphinxstylestrong{RAW} в меню фотосъемки \sphinxincludegraphics[width=25\sphinxpxdimen]{{icon_cam}.png} (Вкладка 1);

\end{itemize}

\begin{figure}[H]
\centering
\capstart

\noindent\sphinxincludegraphics[width=300\sphinxpxdimen]{{menu10}.png}
\caption{Установка качества}\label{\detokenize{camera:id10}}\end{figure}
\begin{itemize}
\item {} 
\sphinxstylestrong{ISO} не более 400 (для настройки нажать колесико управления вправо);

\item {} 
\sphinxstylestrong{Коррекция экспозиции} от EV +1 до EV +2 (для настройки нажать колесико управления вниз).

\end{itemize}


\section{Настройка камеры Sony DSC-RX1RM2}
\label{\detokenize{camera:sony-dsc-rx1rm2}}
Перед выполнением настроек ознакомьтесь с \sphinxhref{https://www.sony.ru/electronics/support/res/manuals/4469/44695786M.pdf}{инструкцией по эксплуатации фотокамеры}, в которой описано назначение и использование органов управления.
\begin{itemize}
\item {} 
Диск режимов установите в положение \sphinxstylestrong{S} (Приоритет выдержки).

\item {} 
Установите выдержку \sphinxstylestrong{1/1000}.

\item {} 
Диск коррекции экспозиции установите в положение \sphinxstylestrong{0} (ноль).

\item {} 
Кольцо переключения макро установите в положение \sphinxstylestrong{0,3m-\(\infty\)}.

\item {} 
Диск режима фокусировки установите в положение \sphinxstylestrong{MF}.

\end{itemize}

Для настройки нужно нажать на кнопку \sphinxstylestrong{MENU}, затем, в соответствии с пунктами ниже, установить требуемые значения.
\begin{itemize}
\item {} 
В меню пользовательских настроек \sphinxincludegraphics[width=25\sphinxpxdimen]{{icon_set}.png} (вкладка 1) отключите \sphinxstylestrong{Автоматический просмотр}.

\end{itemize}

\begin{figure}[H]
\centering
\capstart

\noindent\sphinxincludegraphics[width=400\sphinxpxdimen]{{menu2}.png}
\caption{Отключение автоматического просмотра}\label{\detokenize{camera:id11}}\end{figure}
\begin{itemize}
\item {} 
Установки для кнопки \sphinxstylestrong{MOVIE} — \sphinxstylestrong{Только режим видео} (вкладка 3).

\end{itemize}

\begin{figure}[H]
\centering
\capstart

\noindent\sphinxincludegraphics[width=400\sphinxpxdimen]{{menu11}.png}
\caption{Установка значения «Только режим видео»}\label{\detokenize{camera:id12}}\end{figure}
\begin{itemize}
\item {} 
В меню настроек \sphinxincludegraphics[width=25\sphinxpxdimen]{{icon_key}.png} (вкладка 2) установите \sphinxstylestrong{Время начала энергосбережения} — \sphinxstylestrong{30 мин}.

\end{itemize}

\begin{figure}[H]
\centering
\capstart

\noindent\sphinxincludegraphics[width=400\sphinxpxdimen]{{menu1}.png}
\caption{Установка времени начала энергосбережения}\label{\detokenize{camera:id13}}\end{figure}
\begin{itemize}
\item {} 
В меню карты памяти \sphinxincludegraphics[width=25\sphinxpxdimen]{{icon_sd}.png} установите \sphinxstylestrong{Номер файла} — \sphinxstylestrong{Сброс}.

\end{itemize}

\begin{figure}[H]
\centering
\capstart

\noindent\sphinxincludegraphics[width=400\sphinxpxdimen]{{menu3}.png}
\caption{Сброс номера файла}\label{\detokenize{camera:id14}}\end{figure}

Другие настройки камеры изменять не требуется.

\sphinxstylestrong{Форматирование карты памяти}
\begin{enumerate}
\def\theenumi{\arabic{enumi}}
\def\labelenumi{\theenumi )}
\makeatletter\def\p@enumii{\p@enumi \theenumi )}\makeatother
\item {} 
Последовательно выберите \sphinxstylestrong{MENU \(\rightarrow\)} \sphinxincludegraphics[width=25\sphinxpxdimen]{{icon_sd}.png} \sphinxstylestrong{{[}Карта памяти{]}} \sphinxstylestrong{\(\rightarrow\) Форматировать}

\end{enumerate}

\begin{sphinxadmonition}{attention}{Внимание:}
Все данные на карте памяти будут удалены!
\end{sphinxadmonition}

\sphinxstylestrong{Сброс настроек}

Для сброса всех настроек:
\begin{enumerate}
\def\theenumi{\arabic{enumi}}
\def\labelenumi{\theenumi )}
\makeatletter\def\p@enumii{\p@enumi \theenumi )}\makeatother
\item {} 
Последовательно выберите \sphinxstylestrong{MENU \(\rightarrow\)} \sphinxincludegraphics[width=25\sphinxpxdimen]{{icon_key}.png} \sphinxstylestrong{{[}Настройки{]} \(\rightarrow\) Инициализировать  \(\rightarrow\) Сброс настроек}

\end{enumerate}

\begin{sphinxadmonition}{attention}{Внимание:}
Не извлекайте батарейный блок во время сброса или преустановки значений настроек!
\end{sphinxadmonition}
\begin{enumerate}
\def\theenumi{\arabic{enumi}}
\def\labelenumi{\theenumi )}
\makeatletter\def\p@enumii{\p@enumi \theenumi )}\makeatother
\setcounter{enumi}{1}
\item {} 
После сброса установите время и дату. Если пропустить это действие, настройки фотокамеры сохраняться не будут.

\item {} 
Выключите камеру с помощью переключателя ON/OFF.

\item {} 
Не отключайте фотокамеру от сети или АКБ в течение 3-х минут для сохранения настроек во внутреннюю память.

\end{enumerate}

\begin{sphinxadmonition}{important}{Важно:}
Фотокамера может выводить сообщение \sphinxstylestrong{E:61:00}. Это означает, что он сфокусирован на максимальную дальность. На работоспособность не влияет.
\end{sphinxadmonition}


\chapter{Зарядое устройство и АКБ}
\label{\detokenize{charger:id1}}\label{\detokenize{charger::doc}}

\begin{savenotes}\sphinxattablestart
\centering
\sphinxcapstartof{table}
\sphinxthecaptionisattop
\sphinxcaption{\sphinxstylestrong{Характеристики АКБ}}\label{\detokenize{charger:id8}}
\sphinxaftertopcaption
\begin{tabulary}{\linewidth}[t]{|T|T|}
\hline

Гарантированное количество циклов заряд-разряд
&
50
\\
\hline
Верхний предел заряда
&
16,8 В
\\
\hline
Нижний предел разряда
&
13,2 В
\\
\hline
Ток заряда
&
\textless{}10 А
\\
\hline
Количество ячеек
&
4
\\
\hline
Емкость
&
10 000 мА·ч
\\
\hline
\end{tabulary}
\par
\sphinxattableend\end{savenotes}


\section{Техника безопасности}
\label{\detokenize{charger:id2}}
\sphinxstylestrong{АКБ}
\begin{itemize}
\item {} 
Не допускайте разгерметизации и деформации элементов АКБ (не ронять, не прокалывать).

\item {} 
Не допускайте нагрева АКБ свыше 60°С.

\item {} 
Не допускайте перезаряда АКБ (свыше 16,8 В); Не допускайте разряда АКБ ниже 12 В.

\item {} 
Не храните АКБ в разряженном состоянии.

\item {} 
Не заряжайте токами, превышающими нагрузочную способность (не более 100\% от емкости, для продления срока службы рекомендуется заряжать 50\% током от емкости). Превышение допустимого тока заряда приведет к нагреву АКБ свыше 60°С.

\item {} 
При длительном хранении (месяц и более) необходимо перевести АКБ в режим \sphinxstylestrong{Хранение}.

\end{itemize}

\begin{sphinxadmonition}{attention}{Внимание:}
Несоблюдение выше перечисленных указаний может привести к полному выходу из строя АКБ или к возгоранию.
\end{sphinxadmonition}

\sphinxstylestrong{Зарядная станция}
\begin{itemize}
\item {} 
Перед подключением АКБ к зарядному устройству необходимо предварительно включить зарядное устройство.

\item {} 
Перед каждым использованием необходимо производить осмотр кабелей и разъемов на предмет повреждений.

\item {} 
Запрещается эксплуатировать зарядное устройство под прямыми солнечными лучами.

\item {} 
Запрещается эксплуатировать зарядное устройство без присмотра.

\end{itemize}

С завода комплекс поставляется с настроенным зарядным устройством. Если настройки сбились - следуйте инструкции, чтобы их восстановить.


\section{Предварительные настройки зарядного устройства}
\label{\detokenize{charger:id3}}
Настройка зарядного устройства осуществляется выбором соответствующего пункта меню с помощью сенсорного экрана. При включении зарядного устройства вы увидите следующий экран:

\begin{figure}[H]
\centering
\capstart

\noindent\sphinxincludegraphics[width=470\sphinxpxdimen]{{charge1}.png}
\caption{Главное меню}\label{\detokenize{charger:id9}}\end{figure}

Если по умолчанию на зарядном устройстве установлен английский язык, зайдите в пункт меню \sphinxstylestrong{Uset}:

\begin{figure}[H]
\centering
\capstart

\noindent\sphinxincludegraphics[width=470\sphinxpxdimen]{{lang1}.png}
\caption{Настройки}\label{\detokenize{charger:id10}}\end{figure}

Выберите \sphinxstylestrong{Language}:

\begin{figure}[H]
\centering
\capstart

\noindent\sphinxincludegraphics[width=400\sphinxpxdimen]{{lang2}.png}
\caption{Язык}\label{\detokenize{charger:id11}}\end{figure}

Установите \sphinxstylestrong{Русский язык}.

\sphinxstylestrong{Настройка параметров зарядки АКБ}

В разделе \sphinxstylestrong{Тип} выберите \sphinxstylestrong{LiPo}:

\begin{figure}[H]
\centering
\capstart

\noindent\sphinxincludegraphics[width=400\sphinxpxdimen]{{charge2}.png}
\caption{Тип}\label{\detokenize{charger:id12}}\end{figure}

В разделе \sphinxstylestrong{Элементы} стрелками выберите \sphinxstylestrong{4Cells} \sphinxstylestrong{14.8V} и подтвердите выбор нажатием \sphinxstylestrong{Оk}:

\begin{figure}[H]
\centering
\capstart

\noindent\sphinxincludegraphics[width=400\sphinxpxdimen]{{charge3}.png}
\caption{Элементы}\label{\detokenize{charger:id13}}\end{figure}

В разделе \sphinxstylestrong{Режим} выберите:
\begin{itemize}
\item {} 
\sphinxstylestrong{Баланс} для заряда АКБ;

\item {} 
\sphinxstylestrong{Хранение} для перевода АКБ в режим хранения;

\item {} 
\sphinxstylestrong{Заряд} для заряда АКБ без балансировки напряжения на элементах (не рекомендуется заряжать в данном режиме);

\item {} 
\sphinxstylestrong{Разряд} для разряда АКБ;

\item {} 
\sphinxstylestrong{Быстр.зар.} для заряда повышенными токами (не рекомендуется заряжать в данном режиме);

\item {} 
\sphinxstylestrong{Проверка} для проверки состояния АКБ.

\end{itemize}

\begin{figure}[H]
\centering
\capstart

\noindent\sphinxincludegraphics[width=400\sphinxpxdimen]{{charge4}.png}
\caption{Режим}\label{\detokenize{charger:id14}}\end{figure}

В разделе \sphinxstylestrong{Ток} выберите \sphinxstylestrong{5.0А} в верхней шкале (ток заряда), \sphinxstylestrong{3.0А} в нижней шкале (ток разряда) и подтвердите выбор нажатием \sphinxstylestrong{Оk}:

\begin{figure}[H]
\centering
\capstart

\noindent\sphinxincludegraphics[width=350\sphinxpxdimen]{{charge5}.png}
\caption{Ток}\label{\detokenize{charger:id15}}\end{figure}

Перейдите в раздел \sphinxstylestrong{Настройки}:

\begin{figure}[H]
\centering
\capstart

\noindent\sphinxincludegraphics[width=350\sphinxpxdimen]{{charge7}.png}
\caption{Настройки}\label{\detokenize{charger:id16}}\end{figure}

В разделе \sphinxstylestrong{Настройки} установите \sphinxstylestrong{Отсечка по времени} \sphinxstylestrong{200 Minute} и подтвердите нажатием \sphinxstylestrong{Оk}:

\begin{figure}[H]
\centering
\capstart

\noindent\sphinxincludegraphics[width=350\sphinxpxdimen]{{charge6}.png}
\caption{Отсечка по времени}\label{\detokenize{charger:id17}}\end{figure}

В разделе \sphinxstylestrong{Настройки} установите \sphinxstylestrong{Отсечка по емкости} \sphinxstylestrong{10.0 Ah} и подтвердите нажатием \sphinxstylestrong{Оk}:

\begin{figure}[H]
\centering
\capstart

\noindent\sphinxincludegraphics[width=400\sphinxpxdimen]{{charge8}.png}
\caption{Отсечка по емкости}\label{\detokenize{charger:id18}}\end{figure}


\section{Зарядка АКБ}
\label{\detokenize{charger:id4}}\begin{enumerate}
\def\theenumi{\arabic{enumi}}
\def\labelenumi{\theenumi )}
\makeatletter\def\p@enumii{\p@enumi \theenumi )}\makeatother
\item {} 
Подключите кабель сети переменного тока к ЗУ.

\item {} 
Вставьте адаптер кабеля в розетку.

\item {} 
Подключите балансировочный кабель к заряжаемой АКБ.

\item {} 
Подключите разъем силового кабеля к заряжаемой АКБ.

\item {} 
Проверьте настройки ЗУ и нажмите \sphinxstylestrong{Старт}, чтобы запустить процесс зарядки.

\item {} 
После окончания заряда отключите АКБ в обратном порядке.

\end{enumerate}

\begin{figure}[H]
\centering
\capstart

\noindent\sphinxincludegraphics[width=400\sphinxpxdimen]{{charge9}.png}
\caption{Старт заряда}\label{\detokenize{charger:id19}}\end{figure}


\section{Рекомендации по использованию литий-полимерной (LiPo) АКБ}
\label{\detokenize{charger:lipo}}
\sphinxstylestrong{Правила эксплуатации АКБ}

Во избежание аварийных ситуаций, связанных с нештатной работой АКБ, необходимо соблюдать ряд следующих правил:
\begin{itemize}
\item {} 
если полеты проходят при температуре воздуха ниже 0°С, перед полетом необходимо хранить АКБ в теплом месте и не допускать ее охлаждения. Следует помнить, что на холоде литий-полимерные АКБ могут терять до 30\% своей емкости, данное обстоятельство необходимо учитывать при построении полетного задания;

\item {} 
если полеты проходят при температуре воздуха выше 25°С, перед полетом необходимо хранить АКБ в прохладном, защищенном от попадания прямых солнечных лучей месте. После полета нельзя сразу заряжать АКБ, необходимо дать ей остыть. Заряжать также необходимо в прохладном, защищенном от попадания прямых солнечных лучей месте.

\end{itemize}


\section{Проверка исправности АКБ}
\label{\detokenize{charger:id5}}
Для проверки исправности состояния АКБ рекомендуется провести полный цикл заряд/разряд для АКБ.
\begin{enumerate}
\def\theenumi{\arabic{enumi}}
\def\labelenumi{\theenumi )}
\makeatletter\def\p@enumii{\p@enumi \theenumi )}\makeatother
\item {} 
Зарядите АКБ.

\item {} 
Разрядите АКБ, сменив режим на \sphinxstylestrong{Разряд}, до напряжения 13,5 В.

\item {} 
Повторно зарядите АКБ.

\end{enumerate}


\section{Хранение и разряд}
\label{\detokenize{charger:id6}}
Режим хранения необходим, если предполагается не использовать АКБ более 14 дней.

Для перевода АКБ в режим хранения необходимо сменить режим \sphinxstylestrong{Баланс} на \sphinxstylestrong{Хранение}, проверив правильность количества ячеек на экране (4 Cells), и запустить его нажатием кнопки \sphinxstylestrong{Старт}.

Хранить в сухом прохладном месте, исключающем воздействие прямых солнечных лучей, при температуре от 5 до 25°С и относительной влажности не более 80\%, без конденсации.

Оптимальная температура — от 5 до 10°С.

Оптимальный уровень напряжения АКБ при помещении батареи на хранение: 15,4 В.

Срок хранения - 1 год.


\section{Утилизация АКБ}
\label{\detokenize{charger:id7}}
\begin{sphinxadmonition}{attention}{Внимание:}
Не выбрасывайте LiPo батареи в контейнеры для бытового мусора.
Неправильная утилизация отработавших источников питания может представлять опасность для окружающей среды.
Утилизируйте LiPo батареи в соответствии с местным законодательством, сдавая их в ближайшие пункты переработки.
\end{sphinxadmonition}


\chapter{Использование Geoscan Planner}
\label{\detokenize{planner:geoscan-planner}}\label{\detokenize{planner::doc}}

\section{Предварительная настройка}
\label{\detokenize{planner:id1}}\begin{enumerate}
\def\theenumi{\arabic{enumi}}
\def\labelenumi{\theenumi )}
\makeatletter\def\p@enumii{\p@enumi \theenumi )}\makeatother
\item {} 
Подключите модем КРЛ к USB порту ноутбука.

\item {} 
Включите бортовое питание БВС.

\item {} 
Запустите программу \sphinxstylestrong{MdmDisp}.

\end{enumerate}

В правом нижнем углу появится пиктограмма антенны и количество подключенных бортов.

\begin{figure}[H]
\centering
\capstart

\noindent\sphinxincludegraphics[width=100\sphinxpxdimen]{{planner1}.png}
\caption{Индикатор работы программы MdmDisp}\label{\detokenize{planner:id14}}\end{figure}

3.1 При первом подключении необходимо настроить соединение с БВС, запустив программу \sphinxstylestrong{NetTopology}:
\begin{itemize}
\item {} 
Нажмите на значок \sphinxstylestrong{Поиск новых устройств}.

\end{itemize}

\begin{figure}[H]
\centering
\capstart

\noindent\sphinxincludegraphics[width=200\sphinxpxdimen]{{planner29}.png}
\caption{Значок поиска новых устройств}\label{\detokenize{planner:id15}}\end{figure}

Программа отобразит список обнаруженных модемов.

\begin{sphinxadmonition}{note}{Примечание:}
Эфир сканируется до тех пор, пока кнопка \sphinxstylestrong{Поиск новых устройств} не будет нажата повторно.
\end{sphinxadmonition}
\begin{itemize}
\item {} 
Выберите появившийся Борт №xxx и нажмите на значок \sphinxstylestrong{Добавить устройство}.

\end{itemize}

\begin{figure}[H]
\centering
\capstart

\noindent\sphinxincludegraphics[width=300\sphinxpxdimen]{{planner30}.png}
\caption{Значок добавления устройства}\label{\detokenize{planner:id16}}\end{figure}

Программа сохраняет список добавленных устройств.

При проведении повторных полетов достаточно запустить \sphinxstylestrong{MdmDisp} и убедиться, что подключение выполнено успешно.

Если БВС не обнаружено, вы можете переподключить наземный модем, нажав на значок \sphinxstylestrong{MdmDisp} правой кнопкой мыши и выбрав \sphinxstylestrong{Переподключить}.

\begin{figure}[H]
\centering
\capstart

\noindent\sphinxincludegraphics[width=150\sphinxpxdimen]{{planner2}.png}
\caption{Контекстное меню MdmDisp}\label{\detokenize{planner:id17}}\end{figure}
\begin{enumerate}
\def\theenumi{\arabic{enumi}}
\def\labelenumi{\theenumi )}
\makeatletter\def\p@enumii{\p@enumi \theenumi )}\makeatother
\setcounter{enumi}{3}
\item {} 
Запустите программу \sphinxstylestrong{Geoscan Planner}.

\item {} 
В окне ввода логина и пароля введите свой логин и пароль пользователя продукта.

\item {} 
Во вкладке \sphinxstylestrong{Полет} выберите \sphinxstylestrong{Подключить БВС - Поиск…}.

\end{enumerate}

\begin{figure}[H]
\centering
\capstart

\noindent\sphinxincludegraphics[width=500\sphinxpxdimen]{{planner3}.png}
\caption{Подключение БВС}\label{\detokenize{planner:id18}}\end{figure}
\begin{enumerate}
\def\theenumi{\arabic{enumi}}
\def\labelenumi{\theenumi )}
\makeatletter\def\p@enumii{\p@enumi \theenumi )}\makeatother
\setcounter{enumi}{6}
\item {} 
Выберите тип подключения \sphinxstylestrong{MdmDisp}. Задайте \sphinxstylestrong{IP-адрес} \sphinxstyleemphasis{localhost}. В списке \sphinxstylestrong{Борт} установите для \sphinxstylestrong{БВС - Порт 6}.

\end{enumerate}

\begin{figure}[H]
\centering
\capstart

\noindent\sphinxincludegraphics[width=500\sphinxpxdimen]{{pl4}.png}
\caption{Окно подключения БВС}\label{\detokenize{planner:id19}}\end{figure}

\begin{sphinxadmonition}{note}{Примечание:}
Параметры достаточно установить один раз. При последующих подключениях БВС воспользуйтесь кнопкой \sphinxstylestrong{Подключить БВС} панели инструментов. Приемник автоматически определит координаты и отобразит местоположение БВС на карте. В окне программы появятся панель телеметрии (слева) и панель приборов (справа).
\end{sphinxadmonition}

\begin{figure}[H]
\centering
\capstart

\noindent\sphinxincludegraphics[width=300\sphinxpxdimen]{{pl5}.png}
\caption{Подключение БВС}\label{\detokenize{planner:id20}}\end{figure}


\section{Проектирование полетного задания}
\label{\detokenize{planner:id2}}\begin{enumerate}
\def\theenumi{\arabic{enumi}}
\def\labelenumi{\theenumi )}
\makeatletter\def\p@enumii{\p@enumi \theenumi )}\makeatother
\item {} 
Создайте \sphinxstylestrong{Новый проект}.

\end{enumerate}

\begin{figure}[H]
\centering
\capstart

\noindent\sphinxincludegraphics[width=230\sphinxpxdimen]{{planner5}.png}
\caption{Создание нового проекта}\label{\detokenize{planner:id21}}\end{figure}
\begin{enumerate}
\def\theenumi{\arabic{enumi}}
\def\labelenumi{\theenumi )}
\makeatletter\def\p@enumii{\p@enumi \theenumi )}\makeatother
\setcounter{enumi}{1}
\item {} 
Укажите имя проекта, параметры съемки, модель БВС и фотоаппарата.

\end{enumerate}

\begin{figure}[H]
\centering
\capstart

\noindent\sphinxincludegraphics[width=500\sphinxpxdimen]{{planner6}.png}
\caption{Создание нового проекта полетного задания}\label{\detokenize{planner:id22}}\end{figure}


\section{Площадная аэрофотосъемка}
\label{\detokenize{planner:id3}}
Площадная аэрофотосъемка \textendash{} съемка полигонов. Полигон \textendash{} это область, ограниченная многоугольником. Оператор задает вершины многоугольника, а программа автоматически рассчитывает маршрут обхода.
\begin{enumerate}
\def\theenumi{\arabic{enumi}}
\def\labelenumi{\theenumi )}
\makeatletter\def\p@enumii{\p@enumi \theenumi )}\makeatother
\item {} 
Нажмите на значок \sphinxstylestrong{Создать площадную аэрофотосъемку} панели инструментов.

\end{enumerate}

\begin{figure}[H]
\centering
\capstart

\noindent\sphinxincludegraphics[width=300\sphinxpxdimen]{{planner8}.png}
\caption{Создание площадной аэрофотосъемки}\label{\detokenize{planner:id23}}\end{figure}
\begin{enumerate}
\def\theenumi{\arabic{enumi}}
\def\labelenumi{\theenumi )}
\makeatletter\def\p@enumii{\p@enumi \theenumi )}\makeatother
\setcounter{enumi}{1}
\item {} 
Задайте на карте угловые точки исследуемого участка местности. Программа автоматически рассчитает маршрут обхода полигона. При построении маршрута, если разница высот соседних точек превышает 30 метров, набор высоты и снижение БВС отображаются в виде цилиндров. Если БВС набирает высоту, то цилиндр залит оранжевым цветом, иначе \textendash{} синим.

\end{enumerate}

\begin{figure}[H]
\centering
\capstart

\noindent\sphinxincludegraphics[width=500\sphinxpxdimen]{{planner9}.png}
\caption{Цилиндры набора высоты и снижения}\label{\detokenize{planner:id24}}\end{figure}


\subsection{Добавление и удаление вершин полигона}
\label{\detokenize{planner:id4}}
В готовый полигон можно добавлять вершины.
\begin{enumerate}
\def\theenumi{\arabic{enumi}}
\def\labelenumi{\theenumi )}
\makeatletter\def\p@enumii{\p@enumi \theenumi )}\makeatother
\item {} 
С зажатой левой кнопкой мыши переместите среднюю точку стороны полигона.

\end{enumerate}

\begin{figure}[H]
\centering
\capstart

\noindent\sphinxincludegraphics[width=400\sphinxpxdimen]{{planner10}.png}
\caption{Добавление вершины}\label{\detokenize{planner:id25}}\end{figure}

Вершина будет создана автоматически.
В плавающем окне рядом с вершиной отобразятся ее координаты.

\begin{figure}[H]
\centering
\capstart

\noindent\sphinxincludegraphics[width=340\sphinxpxdimen]{{planner34}.png}
\caption{Результат добавления вершины}\label{\detokenize{planner:id26}}\end{figure}

Для удаления вершины:
\begin{enumerate}
\def\theenumi{\arabic{enumi}}
\def\labelenumi{\theenumi )}
\makeatletter\def\p@enumii{\p@enumi \theenumi )}\makeatother
\item {} 
нажмите на вершину правой кнопкой мыши;

\item {} 
в контекстном меню выберите \sphinxstylestrong{Удалить вершину}.

\end{enumerate}

\begin{figure}[H]
\centering
\capstart

\noindent\sphinxincludegraphics[width=330\sphinxpxdimen]{{planner33}.png}
\caption{Удаление вершины}\label{\detokenize{planner:id27}}\end{figure}


\subsection{Изменение направления линий облета}
\label{\detokenize{planner:id5}}
Необходимость оптимизировать полигон «по направлению» возникает, например, если на месте проведения работ сила и направление ветра неблагоприятны (сильный ветер вдоль линий облета полигона).
Для изменения типа оптимизации щелкните правой кнопкой мыши на полигоне и выберите в контекстном меню вариант \sphinxstylestrong{Оптимизация «направление»}.
\begin{enumerate}
\def\theenumi{\arabic{enumi}}
\def\labelenumi{\theenumi )}
\makeatletter\def\p@enumii{\p@enumi \theenumi )}\makeatother
\item {} 
Нажмите правой кнопкой мыши по области полигона.

\item {} 
В контекстном меню выберите \sphinxstylestrong{Оптимизация «направление»}.

\end{enumerate}

\begin{figure}[H]
\centering
\capstart

\noindent\sphinxincludegraphics[width=450\sphinxpxdimen]{{planner11}.png}
\caption{Оптимизация по направлению}\label{\detokenize{planner:id28}}\end{figure}

Одна из вершин полигона будет подсвечена, на ней появится маркер поворота для задания направления.

\begin{figure}[H]
\centering
\capstart

\noindent\sphinxincludegraphics[width=450\sphinxpxdimen]{{planner12}.png}
\caption{Корректировка направления облета}\label{\detokenize{planner:id29}}\end{figure}

Результатом будет новый маршрут облета полигона по заданному направлению.

\begin{figure}[H]
\centering
\capstart

\noindent\sphinxincludegraphics[width=450\sphinxpxdimen]{{planner13}.png}
\caption{Новый маршрут облета}\label{\detokenize{planner:id30}}\end{figure}


\subsection{Изменение точки входа}
\label{\detokenize{planner:id6}}
Для изменения точки входа в полигон выполните следующие действия:
\begin{enumerate}
\def\theenumi{\arabic{enumi}}
\def\labelenumi{\theenumi )}
\makeatletter\def\p@enumii{\p@enumi \theenumi )}\makeatother
\item {} 
Выделите полигон.

\end{enumerate}

\begin{figure}[H]
\centering
\capstart

\noindent\sphinxincludegraphics[width=450\sphinxpxdimen]{{planner14}.png}
\caption{Выделенный полигон}\label{\detokenize{planner:id31}}\end{figure}
\begin{enumerate}
\def\theenumi{\arabic{enumi}}
\def\labelenumi{\theenumi )}
\makeatletter\def\p@enumii{\p@enumi \theenumi )}\makeatother
\setcounter{enumi}{1}
\item {} 
Правой кнопкой мыши выделите точку, в которой нужно осуществить вход.

\item {} 
В появившемся контекстном меню выберите \sphinxstylestrong{Начать здесь}.

\end{enumerate}

\begin{figure}[H]
\centering
\capstart

\noindent\sphinxincludegraphics[width=500\sphinxpxdimen]{{planner15}.png}
\caption{Изменение точки входа в полигон}\label{\detokenize{planner:id32}}\end{figure}

У выбранной точки входа появится флажок \sphinxincludegraphics[width=50\sphinxpxdimen]{{flag}.png}


\section{Линейная аэрофотосъемка}
\label{\detokenize{planner:id7}}
Линейная аэрофотосъемка - облет линейных протяженных объектов, таких как: реки, дороги, нефтепроводы и т.п..
\begin{enumerate}
\def\theenumi{\arabic{enumi}}
\def\labelenumi{\theenumi )}
\makeatletter\def\p@enumii{\p@enumi \theenumi )}\makeatother
\item {} 
Нажмите на значок \sphinxstylestrong{Создать линейную аэрофотосъемку} панели инструментов.

\end{enumerate}

\begin{figure}[H]
\centering
\capstart

\noindent\sphinxincludegraphics[width=300\sphinxpxdimen]{{planner16}.png}
\caption{Создание области линейной аэрофотосъемки}\label{\detokenize{planner:id33}}\end{figure}
\begin{enumerate}
\def\theenumi{\arabic{enumi}}
\def\labelenumi{\theenumi )}
\makeatletter\def\p@enumii{\p@enumi \theenumi )}\makeatother
\setcounter{enumi}{1}
\item {} 
Однократными щелчками задайте маршрут обхода протяженного объекта по точкам разворотов. Программа автоматически построит линии облета.

\end{enumerate}

\begin{figure}[H]
\centering
\capstart

\noindent\sphinxincludegraphics[width=500\sphinxpxdimen]{{planner17}.png}
\caption{Пример линейной аэрофотосъемки}\label{\detokenize{planner:id34}}\end{figure}


\subsection{Изменение параметров БВС в точках разворота}
\label{\detokenize{planner:id8}}
По умолчанию поведение БВС в точках разворота выбирается автоматически с учетом угла между соседними линиями (развороты на углы до заданного угла автопролета осуществляются пролетом).
\begin{enumerate}
\def\theenumi{\arabic{enumi}}
\def\labelenumi{\theenumi )}
\makeatletter\def\p@enumii{\p@enumi \theenumi )}\makeatother
\item {} 
Нажмите правой кнопкой мыши на вершину.

\item {} 
В появившемся контекстном меню выберите необходимый тип осуществления разворота.

\end{enumerate}

\begin{figure}[H]
\centering
\capstart

\noindent\sphinxincludegraphics[width=500\sphinxpxdimen]{{planner18}.png}
\caption{Настройка параметров прохождения вершины}\label{\detokenize{planner:id35}}\end{figure}
\begin{itemize}
\item {} 
\sphinxstylestrong{Разворот с выходом на ЛЗП} \sphinxstyleemphasis{(линия заданного пути)} означает, что БВС полностью пролетит галс, а затем зайдет на следующий галс с дополнительным маневром («петлей»). Этот вариант гарантирует съемку территории под маршрутом в полном объеме, и он предпочтителен в случае резких разворотов.

\item {} 
\sphinxstylestrong{Разворот пролетом} может с успехом применяться при съемке рек и других естественных объектов, не имеющих выраженных точек разворота. Это более быстрый способ разворота, но он плохо подходит для резких разворотов (крайние части территории под линиями маршрута могут оказаться вне зоны съемки).

\end{itemize}


\section{Перелет}
\label{\detokenize{planner:id9}}
Добавление перелетов в полетное задание необходимо, если в зоне полета могут оказаться точечные высотные объекты.
\begin{enumerate}
\def\theenumi{\arabic{enumi}}
\def\labelenumi{\theenumi )}
\makeatletter\def\p@enumii{\p@enumi \theenumi )}\makeatother
\item {} 
Нажмите на значок \sphinxstylestrong{Создать перелет} панели инструментов.

\end{enumerate}

\begin{figure}[H]
\centering
\capstart

\noindent\sphinxincludegraphics[width=300\sphinxpxdimen]{{planner19}.png}
\caption{Создание перелета}\label{\detokenize{planner:id36}}\end{figure}
\begin{enumerate}
\def\theenumi{\arabic{enumi}}
\def\labelenumi{\theenumi )}
\makeatletter\def\p@enumii{\p@enumi \theenumi )}\makeatother
\setcounter{enumi}{1}
\item {} 
Однократными щелчками задайте маршрут перелета. Для построения перелета на разных высотах, выберите в окне «Свойства» \sphinxstylestrong{Режим высоты точек - Нефиксированный}.

\end{enumerate}

\begin{figure}[H]
\centering
\capstart

\noindent\sphinxincludegraphics[width=500\sphinxpxdimen]{{planner20}.png}
\caption{Свойства перелета}\label{\detokenize{planner:id37}}\end{figure}
\begin{itemize}
\item {} 
Функция \sphinxstylestrong{Выполнять фотографирование} активирует работу фотоаппарата.

\end{itemize}

Шаг фотографирования в метрах указывается в соответствующем поле.
Значения в столбце \sphinxstylestrong{Превышение} - это разность абсолютной высоты точки ПЗ и рельефа под ней. Таким образом, высота рельефа обязательно учитывается. Абсолютные высоты точек также доступны для редактирования через столбец \sphinxstylestrong{Высота}. Кроме этого, высоту можно изменять визуальным редактированием (потянуть мышкой с нажатой клавишей \sphinxstyleemphasis{Shift}).

Маршрут перелета между двумя полетными элементами строится по следующим правилам:
\begin{enumerate}
\def\theenumi{\arabic{enumi}}
\def\labelenumi{\theenumi )}
\makeatletter\def\p@enumii{\p@enumi \theenumi )}\makeatother
\item {} 
Если у полетных элементов одинаковая высота, то перелет будет на этой же высоте.

\item {} 
Если у полетных элементов разные высоты, то перелет будет на наибольшей из двух высот.

\end{enumerate}

\begin{sphinxadmonition}{attention}{Внимание:}
Если условия не позволяют достичь высоты второй точки по прямой (например, небольшое расстояние между точками, но большая разница высот), БВС полетит с максимально допустимым тангажом по прямой до достижения заданной точки по координатам, после чего наберет/сбросит высоту по спирали.
\end{sphinxadmonition}


\section{Точка ожидания}
\label{\detokenize{planner:id10}}
Команда \sphinxstylestrong{Создать точку ожидания} служит для удержания БВС на высоте в течение отрезка времени.
\begin{enumerate}
\def\theenumi{\arabic{enumi}}
\def\labelenumi{\theenumi )}
\makeatletter\def\p@enumii{\p@enumi \theenumi )}\makeatother
\item {} 
Нажмите на значок \sphinxstylestrong{Создать точку ожидания} на панели инструментов.

\end{enumerate}

\begin{figure}[H]
\centering
\capstart

\noindent\sphinxincludegraphics[width=300\sphinxpxdimen]{{planner22}.png}
\caption{Создание точки ожидания}\label{\detokenize{planner:id38}}\end{figure}
\begin{enumerate}
\def\theenumi{\arabic{enumi}}
\def\labelenumi{\theenumi )}
\makeatletter\def\p@enumii{\p@enumi \theenumi )}\makeatother
\setcounter{enumi}{1}
\item {} 
Щелчком мыши на карте задайте точку, в которой должно осуществляться ожидание.

\end{enumerate}

В экспертном режиме можно изменить свойства: задать высоту точки ожидания, длительность ожидания, направление движения и активировать функции измерения ветра и бесконечного ожидания.

\begin{figure}[H]
\centering
\capstart

\noindent\sphinxincludegraphics[width=300\sphinxpxdimen]{{planner23}.png}
\caption{Свойства точки ожидания}\label{\detokenize{planner:id39}}\end{figure}

Планер будет на заданной высоте «удерживать» точку в течение указанного времени (по умолчанию 300 секунд), после чего отправится по запланированному маршруту.

При активации варианта \sphinxstylestrong{Измерение ветра} длительность автоматически выставляется в значение 0. При этом точка ожидания окрасится в желтый цвет. Самолет выполняет полный оборот с постоянным измерением ветра.

\begin{figure}[H]
\centering
\capstart

\noindent\sphinxincludegraphics[width=400\sphinxpxdimen]{{planner24}.png}
\caption{Точка измерения ветра}\label{\detokenize{planner:id40}}\end{figure}

Функция бесконечного ожидания служит для постоянного удержания точки (пока не сработает отказ по низкому заряду АКБ, приводящий к автоматическому возврату). При этом цвет точки ожидания сменяется на темно-синий.

\begin{figure}[H]
\centering
\capstart

\noindent\sphinxincludegraphics[width=400\sphinxpxdimen]{{planner25}.png}
\caption{Точка бесконечного ожидания}\label{\detokenize{planner:id41}}\end{figure}

\begin{sphinxadmonition}{attention}{Внимание:}
Рекомендуется устанавливать точку ожидания с измерением ветра перед каждым полетным элементом на высоте полетного элемента. Автопилот, учитывая измеренные данные о ветре, будет плавнее идти по маршруту.
\end{sphinxadmonition}


\section{Маршрут посадки}
\label{\detokenize{planner:id11}}
Команда \sphinxstylestrong{Создать посадку} служит для построения маршрута посадки.

Это обязательное действие при построении полетного задания.

На месте проведения полета определите направление ветра, скорректируйте при необходимости зону полета и выберите место посадки.
Для посадки следует выбирать открытое сухое пространство без деревьев и прочих препятствий.
Площадка для посадки должна быть ровной, желательно с травяным покровом.
\begin{enumerate}
\def\theenumi{\arabic{enumi}}
\def\labelenumi{\theenumi )}
\makeatletter\def\p@enumii{\p@enumi \theenumi )}\makeatother
\item {} 
Нажмите на значок \sphinxstylestrong{Создать посадку} на панели инструментов.

\end{enumerate}

\begin{figure}[H]
\centering
\capstart

\noindent\sphinxincludegraphics[width=230\sphinxpxdimen]{{planner31}.png}
\caption{Создание посадки}\label{\detokenize{planner:id42}}\end{figure}
\begin{enumerate}
\def\theenumi{\arabic{enumi}}
\def\labelenumi{\theenumi )}
\makeatletter\def\p@enumii{\p@enumi \theenumi )}\makeatother
\setcounter{enumi}{1}
\item {} 
Щелчком мыши на карте выберите сначала точку посадки, затем точку захода на посадку.

\end{enumerate}

Программа автоматически создаст маршрут посадки из трех точек (промежуточная точка создается автоматически).

\begin{figure}[H]
\centering
\capstart

\noindent\sphinxincludegraphics[width=500\sphinxpxdimen]{{planner32}.png}
\caption{Пример посадки}\label{\detokenize{planner:id43}}\end{figure}

\begin{sphinxadmonition}{attention}{Внимание:}
Важно, чтобы посадка осуществлялась против ветра в области посадки. В противном случае возможна жесткая посадка, приводящая к повреждениям самолета.
\end{sphinxadmonition}


\section{Предстартовая подготовка}
\label{\detokenize{planner:id12}}\begin{enumerate}
\def\theenumi{\arabic{enumi}}
\def\labelenumi{\theenumi )}
\makeatletter\def\p@enumii{\p@enumi \theenumi )}\makeatother
\item {} 
Запустите \sphinxstylestrong{Мастер предстартовой подготовки}.

\end{enumerate}

\begin{figure}[H]
\centering
\capstart

\noindent\sphinxincludegraphics[width=300\sphinxpxdimen]{{planner26}.png}
\caption{Запуск мастера предстартовой подготовки}\label{\detokenize{planner:id44}}\end{figure}

Следуйте указаниям мастера предстартовой подготовки (большинство проверок выполняются автоматически).
Задайте радиус автоматического отцепа парашюта и время автономного полета (время, в течение которого осуществляется полет независимо от наличия связи между НСУ и БВС).
После прохождения предстартовой подготовки установите БВС на пусковую установку.


\section{Полет}
\label{\detokenize{planner:id13}}\begin{enumerate}
\def\theenumi{\arabic{enumi}}
\def\labelenumi{\theenumi )}
\makeatletter\def\p@enumii{\p@enumi \theenumi )}\makeatother
\item {} 
Нажмите на значок \sphinxstylestrong{Старт}.

\end{enumerate}

\begin{figure}[H]
\centering
\capstart

\noindent\sphinxincludegraphics[width=300\sphinxpxdimen]{{planner27}.png}
\caption{Перевод БВС в стартовый режим}\label{\detokenize{planner:id45}}\end{figure}

БВС перейдет в стартовый режим.
На панели телеметрии отобразится режим \sphinxstylestrong{КАТАПУЛЬТА}.

\begin{figure}[H]
\centering
\capstart

\noindent\sphinxincludegraphics[width=350\sphinxpxdimen]{{planner28}.png}
\caption{Режим КАТАПУЛЬТА}\label{\detokenize{planner:id46}}\end{figure}

\begin{sphinxadmonition}{attention}{Внимание:}
Переводить БВС в стартовый режим необходимо после установки на пусковую установку. После перехода в стартовый режим запрещается брать в руки и переносить БВС.
\end{sphinxadmonition}

\begin{sphinxadmonition}{attention}{Внимание:}
Чтобы отменить переход в режим Катапульта, нажмите кнопку \sphinxstylestrong{Возврат}. БВС перейдет в режим ПОДГОТОВКА. Мастер предстартовой подготовки необходимо будет пройти заново.
\end{sphinxadmonition}
\begin{enumerate}
\def\theenumi{\arabic{enumi}}
\def\labelenumi{\theenumi )}
\makeatletter\def\p@enumii{\p@enumi \theenumi )}\makeatother
\setcounter{enumi}{1}
\item {} 
Снимите предохранитель и активируйте пусковую установку, потянув за спусковой шнур.

\end{enumerate}

БВС осуществит взлет.


\chapter{Запуск}
\label{\detokenize{launch:id1}}\label{\detokenize{launch::doc}}
Чтобы запустить Геоскан Lite, последовательно выполните пункты инструкции:
\begin{enumerate}
\def\theenumi{\arabic{enumi}}
\def\labelenumi{\theenumi )}
\makeatletter\def\p@enumii{\p@enumi \theenumi )}\makeatother
\item {} 
Зарядите АКБ и убедитесь, что она исправно работает (см. {\hyperref[\detokenize{charger::doc}]{\sphinxcrossref{\DUrole{doc}{Зарядое устройство и АКБ}}}}, раздел \sphinxstylestrong{Проверка исправности АКБ}).

\item {} 
Спланируйте полетное задание, используя Geoscan Planner (см. {\hyperref[\detokenize{planner::doc}]{\sphinxcrossref{\DUrole{doc}{Использование Geoscan Planner}}}}). Обязательно задайте маршрут посадки. Сохраните полетное задание, чтобы быстро загрузить его перед вылетом.

\item {} 
Откройте транспортировочную сумку и разверните НСУ.

\item {} 
Соберите пусковую установку (см. {\hyperref[\detokenize{catapult::doc}]{\sphinxcrossref{\DUrole{doc}{Пусковая установка}}}}). Непосредственно перед запуском установите резиновые жгуты и натяните их.

\item {} 
Соберите БВС (см. {\hyperref[\detokenize{uav::doc}]{\sphinxcrossref{\DUrole{doc}{БВС}}}}). Установите парашют.

\end{enumerate}

\begin{sphinxadmonition}{attention}{Внимание:}
Проверьте, что крышка парашютного отсека свободно открывается и закрывается. Для этого поверните поводок машинки отцепа в сторону и приподнимите крышку вверх. Стропа крепления крышки к парашюту не должна западать или цепляться за подкрепляющий штырь крышки. Убедитесь, что крышка свободно открывается и закрывается, а купол сложенного парашюта не попадает в места прилегания крышки к фюзеляжу. Закройте крышку, прижав ее и сдвинув поводок машинки отцепа.
\end{sphinxadmonition}

Закройте крышку парашютного отсека, прижав ее и сдвинув поводок машинки отцепа.
\begin{enumerate}
\def\theenumi{\arabic{enumi}}
\def\labelenumi{\theenumi )}
\makeatletter\def\p@enumii{\p@enumi \theenumi )}\makeatother
\setcounter{enumi}{5}
\item {} 
Отформатируйте карту памяти в фотоаппарате. Подключите разъем фотоаппарата к БВС. Включите фотоаппарат. Настройте фотоаппарат (см. {\hyperref[\detokenize{camera::doc}]{\sphinxcrossref{\DUrole{doc}{Настройка фотокамеры}}}}). Установите фотоаппарат в ложемент.

\item {} 
Установите и подключите АКБ в БВС.

\item {} 
Закройте крышку фюзеляжа БВС и застегните три фиксатора. Снимите крышку объектива фотоаппарата.

\item {} 
Откройте полетное задание в Geoscan Planner. Запустите мастер предстартовой подготовки. Убедитесь в успешном прохождении предстартовой подготовки.

\item {} 
Натяните жгуты пусковой установки. Установите БВС на каретку.

\item {} 
Нажмите на кнопку \sphinxstylestrong{Старт} в окне Geoscan Planner, чтобы перевести БВС в режим «Катапульта». Снимите предохранитель и потяните за спусковой шнур, чтобы запустить БВС.

\end{enumerate}

БВС осуществит взлет.

12. После запуска БВС ослабьте натяжение резиновых жгутов. Для этого одной рукой придерживайте ручку лебедки, а второй - снимите стопор лебедки и плавно размотайте натяжной трос.
После этого жгуты можно снять с пусковой установки, отсоединив карабин от троса и петлю от каретки.


\chapter{Порядок разборки БВС}
\label{\detokenize{launch:id2}}\begin{enumerate}
\def\theenumi{\arabic{enumi}}
\def\labelenumi{\theenumi )}
\makeatletter\def\p@enumii{\p@enumi \theenumi )}\makeatother
\item {} 
Снимите верхнюю крышку фюзеляжа. Для этого отстегните резиновые фиксаторы на носовой части фюзеляжа, затем извлеките заднюю часть крышки из пазов.

\item {} 
Отключите питание.

\end{enumerate}

Карта памяти также может быть извлечена из фотоаппарата для обработки результатов съемки.
\begin{enumerate}
\def\theenumi{\arabic{enumi}}
\def\labelenumi{\theenumi )}
\makeatletter\def\p@enumii{\p@enumi \theenumi )}\makeatother
\setcounter{enumi}{2}
\item {} 
Закройте верхнюю крышку фюзеляжа, вставив заднюю часть крышки в пазы и застегнув резиновые фиксаторы.

\item {} 
Сложите парашют гармошкой и аккуратно уложите в парашютный отсек.

\item {} 
Закройте крышку парашютного отсека, сдвинув в сторону поводок машинки отцепа.

\end{enumerate}

БВС может быть перемещено на подставку для осуществления разборки.
\begin{enumerate}
\def\theenumi{\arabic{enumi}}
\def\labelenumi{\theenumi )}
\makeatletter\def\p@enumii{\p@enumi \theenumi )}\makeatother
\setcounter{enumi}{5}
\item {} 
Уложите БВС на подставку.

\item {} 
Отсоедините кили, оттянув магнитные фиксаторы.

\item {} 
Снимите верхнюю крышку фюзеляжа.

\item {} 
Извлеките АКБ, отстегнув текстильную застежку.

\item {} 
Отсоедините кабельные сборки консолей от автопилота.

\item {} 
Снимите консоли крыла, оттянув их в стороны.

\item {} 
Вытяните соединительный штырь из фюзеляжа.

\item {} 
Закройте верхнюю крышку фюзеляжа, вставив заднюю часть крышки в пазы и застегнув резиновые фиксаторы.

\item {} 
Уложите детали в транспортировочную сумку БВС.

\end{enumerate}



\renewcommand{\indexname}{Алфавитный указатель}
\printindex
\end{document}